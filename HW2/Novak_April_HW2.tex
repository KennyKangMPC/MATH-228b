\documentclass[10pt]{article}
\usepackage[letterpaper]{geometry}
\geometry{verbose,tmargin=1in,bmargin=1in,lmargin=1in,rmargin=1in}
\usepackage{setspace}
\usepackage{ragged2e}
\usepackage{color}
\usepackage{titlesec}
\usepackage{graphicx}
\usepackage{float}
\usepackage{mathtools}
\usepackage{amsmath}
\usepackage[font=small,labelfont=bf,labelsep=period]{caption}
\usepackage[english]{babel}
\usepackage{indentfirst}
\usepackage{array}
\usepackage{makecell}
\usepackage[usenames,dvipsnames]{xcolor}
\usepackage{multirow}
\usepackage{tabularx}
\usepackage{arydshln}
\usepackage{caption}
\usepackage{subcaption}
\usepackage{xfrac}
\usepackage{etoolbox}
\usepackage{cite}
\usepackage{url}
\usepackage{dcolumn}
\usepackage{hyperref}
\usepackage{courier}
\usepackage{url}
\usepackage{esvect}
\usepackage{commath}
\usepackage{verbatim} % for block comments
\usepackage{enumitem}
\usepackage{hyperref} % for clickable table of contents
\usepackage{braket}
\usepackage{titlesec}
\usepackage{booktabs}
\usepackage{gensymb}
\usepackage{longtable}
\usepackage{listings}
\usepackage{cancel}
\usepackage{tcolorbox}
\usepackage[mathscr]{euscript}
\lstset{
    frame=single,
    breaklines=true,
    postbreak=\raisebox{0ex}[0ex][0ex]{\ensuremath{\color{red}\hookrightarrow\space}}
}

% for circled numbers
\usepackage{tikz}
\newcommand*\circled[1]{\tikz[baseline=(char.base)]{
            \node[shape=circle,draw,inner sep=2pt] (char) {#1};}}

\newcommand{\beq}{\begin{equation}}
\newcommand{\eeq}{\end{equation}}
\newcommand{\beqa}{\begin{equation}\begin{aligned}}
\newcommand{\eeqa}{\end{aligned}\end{equation}}

\titleclass{\subsubsubsection}{straight}[\subsection]

% define new command for triple sub sections
\newcounter{subsubsubsection}[subsubsection]
\renewcommand\thesubsubsubsection{\thesubsubsection.\arabic{subsubsubsection}}
\renewcommand\theparagraph{\thesubsubsubsection.\arabic{paragraph}} % optional; useful if paragraphs are to be numbered

\titleformat{\subsubsubsection}
  {\normalfont\normalsize\bfseries}{\thesubsubsubsection}{1em}{}
\titlespacing*{\subsubsubsection}
{0pt}{3.25ex plus 1ex minus .2ex}{1.5ex plus .2ex}

\makeatletter
\renewcommand\paragraph{\@startsection{paragraph}{5}{\z@}%
  {3.25ex \@plus1ex \@minus.2ex}%
  {-1em}%
  {\normalfont\normalsize\bfseries}}
\renewcommand\subparagraph{\@startsection{subparagraph}{6}{\parindent}%
  {3.25ex \@plus1ex \@minus .2ex}%
  {-1em}%
  {\normalfont\normalsize\bfseries}}
\def\toclevel@subsubsubsection{4}
\def\toclevel@paragraph{5}
\def\toclevel@paragraph{6}
\def\l@subsubsubsection{\@dottedtocline{4}{7em}{4em}}
\def\l@paragraph{\@dottedtocline{5}{10em}{5em}}
\def\l@subparagraph{\@dottedtocline{6}{14em}{6em}}
\makeatother

\newcommand{\volume}{\mathop{\ooalign{\hfil$V$\hfil\cr\kern0.08em--\hfil\cr}}\nolimits}

\setcounter{secnumdepth}{4}
\setcounter{tocdepth}{4}
\begin{document}

\title{MATH 228b: HW2 ... 1, 2, 3, 4}
\author{April Novak}

\maketitle

\section{}

The solution of the following problem with linear Lagrange elements for \(u(0)=0\) and \(u^{'}(1)=g\) is equivalent to a second order central difference approximation of the problem. 

\beq
\label{eq:1}
-u''(x)=f(x)
\eeq

The function space for the weight function \(v\) is in the space \(V_h\) such that \(v\) is \(C^0\) continuous over the entire domain (the zeroth derivative is the highest continuous derivative), where over each element \(K\), \(v\) is a linear function. \(v\) satisfies the homogeneous form of the Dirichlet boundary condition.

\beq
V_h=\{v\in C^0([0,1]): v\rvert_K\in P_1(K) \forall K\in T_h, v(0)=0\}
\eeq

The weak form is obtained by multiplying Eq. \eqref{eq:1} by \(v\) and integrating over the domain:

\beq
-\int_0^1 u''(x)v(x)dx=\int_0^1 f(x)v(x)dx
\eeq

The first term on the left-hand side can be rewritten using the divergence rule:

\beq
\int_0^1 u'(x)v'(x)dx-\left\lbrack u'(x)v(x)\right\rbrack_0^1=\int_0^1 f(x)v(x)dx
\eeq

At this point, the weak form contains no information about the boundary conditions. By applying the boundary conditions, the above reduces to:

\beq
\int_0^1 u'(x)v'(x)dx=\int_0^1 f(x)v(x)dx+gv(1)
\eeq

Using a linear Lagrange basis, defined on a mesh with element sizes \(h\), the elements are defined on \(e^1: [0, h]; e^2: [h, 2h]; e^3: [3h, 4h]; \cdots; e^n: [1-h, h]\), where \(n\) is the number of elements. The left and right coordinates of each element will be denoted as \(x_i\) and \(x_{i+1}\) for simplicity. Then, over an arbitrary element, the linear Lagrange shape functions \(\psi\) are:

\beqa
\psi_i(x)=& \frac{x_{i+1}-x}{x_{i+1}-x_i}=\frac{x_{i+1}-x}{h}\\
\psi_{i+1}(x) =& \frac{x-x_i}{x_{i+1}-x_i}=\frac{x-x_i}{h}\\
\eeqa

And their derivatives with respect to \(x\) are:

\beqa
\psi_i'(x)=&\frac{-1}{h}\\
\psi_{i+1}'(x) =&\frac{1}{h}\\
\eeqa

For equal-sized elements without edges on the Neumann boundary, the element stiffness matrix and element load vector are given as:

\beqa
A_{ij}^k=&\int_{x_i}^{x_{i+1}}\psi_i'(x)v_j'(x)dx\\
f_j^k=&\int_{x_i}^{x_{i+1}}f(x)v_j(x)dx\\
\eeqa

By inserting the shape functions given above, \(\textbf{A}^k\) becomes:

\beqa
\textbf{A}^k=& \begin{bmatrix}	\int_{x_i}^{x_{i+1}}\frac{-1}{h}\frac{-1}{h}dx & 
										\int_{x_i}^{x_{i+1}}\frac{-1}{h}\frac{1}{h}dx\\
										\int_{x_i}^{x_{i+1}}\frac{1}{h}\frac{-1}{h}dx &
										\int_{x_i}^{x_{i+1}}\frac{1}{h}\frac{1}{h}dx
					\end{bmatrix}\\
			   =& \begin{bmatrix}	\frac{1}{h} & 
										\frac{-1}{h}\\
										\frac{-1}{h} &
										\frac{1}{h}
					\end{bmatrix}
\eeqa

And the load vector becomes, with \(f=1\) chosen for simplicity:

\beqa
\textbf{f}^{\ k}=& \begin{bmatrix}	\int_{x_i}^{x_{i+1}}\frac{x_{i+1}-x}{h}dx \\
										\int_{x_i}^{x_{i+1}}\frac{x-x_i}{h}dx
					\end{bmatrix}\\
			   =& \begin{bmatrix}	\frac{(x_{i+1}-x_i)^2}{2h}\\
										\frac{(x_{i+1}-x_i)^2}{2h}
					\end{bmatrix}\\
				=& \begin{bmatrix}	h/2\\ 
										h/2
					\end{bmatrix}
\eeqa

Then, by the connectivity matrix relating how the nodes touch each other, the above matrices are ``tesselated'' along the diagonal of the global \textbf{A} and \textbf{f}. Suppose for the time being that \(u(1)=0\), i.e. both ends have homogeneous Dirichlet conditions. Then, the matrix system before application of the Dirichlet conditions is:

\beqa
\label{eq:3}
\begin{bmatrix}
1/h & -1/h & & & & &\\
-1/h & 2/h & -1/h & & & & \\
 & -1/h & 2/h & -1/h & & & \\
 & & -1/h & 2/h & -1/h & & \\
 & & & & & &\\
\end{bmatrix}
\begin{bmatrix}u_1\\u_2\\\vdots\\u_{n+1}\end{bmatrix} = 
\begin{bmatrix}
h/2\\ h\\h\\\vdots\\h\\h/2
\end{bmatrix}
\eeqa

Now, to apply the Neumann condition at \(x=1\), the load vector for the last element (the element spanning \(1-h\leq x\leq1\) must be modified:

\beqa
\textbf{f}^{\ e=n}=& \begin{bmatrix}	\int_{x_i}^{x_{i+1}}\frac{x_{i+1}-x}{h}dx \\
										\int_{x_i}^{x_{i+1}}\frac{x-x_i}{h}dx+g
					\end{bmatrix}\\
			   =& \begin{bmatrix}	\frac{(x_{i+1}-x_i)^2}{2h}\\
										\frac{(x_{i+1}-x_i)^2}{2h}+g
					\end{bmatrix}\\
				=& \begin{bmatrix}	h/2\\ 
										h/2+g
					\end{bmatrix}
\eeqa

Then, the last equation in Eq. \eqref{eq:3} must be modified. To strongly apply the homogeneous Dirichlet condition at \(x=0\), the first row and column of \textbf{A} is eliminated, and post-processing will assign the value \(u_1=0\). The full matrix system, with boundary conditions \(u(0)=0, u'(1)=g\), is, for a four-element system (purely for illustration):

\beqa
\label{eq:3}
\begin{bmatrix}
 2/h & -1/h & 0 & 0 \\
 -1/h & 2/h & -1/h & 0 \\
 0 & -1/h & 2/h & -1/h\\
 0 & 0 & -1/h & 1/h\\
\end{bmatrix}
\begin{bmatrix}u_2\\u_3\\u_4\\u_5\end{bmatrix} = 
\begin{bmatrix}
h\\h\\h\\h/2+g
\end{bmatrix}
\eeqa

Divide both sides by \(h\) for future insight into our choice of \(f(x)=1\):

\beqa
\label{eq:3}\frac{1}{h^2}
\begin{bmatrix}
 2 & -1 & 0 & 0 \\
 -1 & 2 & -1 & 0 \\
 0 & -1 & 2 & -1\\
 0 & 0 & -1 & 1\\
\end{bmatrix}
\begin{bmatrix}u_2\\u_3\\u_4\\u_5\end{bmatrix} = 
\begin{bmatrix}
1\\1\\1\\(h/2+g)/h
\end{bmatrix}
\eeqa

The central difference approximation to a second derivative is:

\beq
u''(x)\approx \frac{u_{i+1}-2u_i+u_{i-1}}{h^2}
\eeq

Looking at the equation for node 3, for instance, we can express this equation for \(i=3\) in terms of the values on either side (\(u_2=u_{i-1}\) and \(u_4=u_{i+1}\)):

\beq
\label{eq:4}
-\frac{1}{h^2}\left(u_{i-1}-2u_i+u_{i+1}\right)=1
\eeq

Eq. \eqref{eq:4} is identical to central difference approximation of the original equation in Eq. \eqref{eq:1} - the second derivative is approximated as a central difference, while the right-hand-side reflects the (arbitrary) choice of setting \(f(x)=1\). The above equation holds for all nodes on the interior. 


\end{document}

