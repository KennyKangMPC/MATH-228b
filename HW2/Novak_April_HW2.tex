\documentclass[10pt]{article}
\usepackage[letterpaper]{geometry}
\geometry{verbose,tmargin=1in,bmargin=1in,lmargin=1in,rmargin=1in}
\usepackage{setspace}
\usepackage{ragged2e}
\usepackage{color}
\usepackage{titlesec}
\usepackage{graphicx}
\usepackage{float}
\usepackage{mathtools}
\usepackage{amsmath}
\usepackage[font=small,labelfont=bf,labelsep=period]{caption}
\usepackage[english]{babel}
\usepackage{indentfirst}
\usepackage{array}
\usepackage{makecell}
\usepackage[usenames,dvipsnames]{xcolor}
\usepackage{multirow}
\usepackage{tabularx}
\usepackage{arydshln}
\usepackage{caption}
\usepackage{subcaption}
\usepackage{xfrac}
\usepackage{etoolbox}
\usepackage{cite}
\usepackage{url}
\usepackage{dcolumn}
\usepackage{hyperref}
\usepackage{courier}
\usepackage{url}
\usepackage{esvect}
\usepackage{commath}
\usepackage{verbatim} % for block comments
\usepackage{enumitem}
\usepackage{hyperref} % for clickable table of contents
\usepackage{braket}
\usepackage{titlesec}
\usepackage{booktabs}
\usepackage{gensymb}
\usepackage{longtable}
\usepackage{listings}
\usepackage{cancel}
\usepackage{tcolorbox}
\usepackage[mathscr]{euscript}
\lstset{
    frame=single,
    breaklines=true,
    postbreak=\raisebox{0ex}[0ex][0ex]{\ensuremath{\color{red}\hookrightarrow\space}}
}

% for circled numbers
\usepackage{tikz}
\newcommand*\circled[1]{\tikz[baseline=(char.base)]{
            \node[shape=circle,draw,inner sep=2pt] (char) {#1};}}

\newcommand{\beq}{\begin{equation}}
\newcommand{\eeq}{\end{equation}}
\newcommand{\beqa}{\begin{equation}\begin{aligned}}
\newcommand{\eeqa}{\end{aligned}\end{equation}}

\titleclass{\subsubsubsection}{straight}[\subsection]

% define new command for triple sub sections
\newcounter{subsubsubsection}[subsubsection]
\renewcommand\thesubsubsubsection{\thesubsubsection.\arabic{subsubsubsection}}
\renewcommand\theparagraph{\thesubsubsubsection.\arabic{paragraph}} % optional; useful if paragraphs are to be numbered

\titleformat{\subsubsubsection}
  {\normalfont\normalsize\bfseries}{\thesubsubsubsection}{1em}{}
\titlespacing*{\subsubsubsection}
{0pt}{3.25ex plus 1ex minus .2ex}{1.5ex plus .2ex}

\makeatletter
\renewcommand\paragraph{\@startsection{paragraph}{5}{\z@}%
  {3.25ex \@plus1ex \@minus.2ex}%
  {-1em}%
  {\normalfont\normalsize\bfseries}}
\renewcommand\subparagraph{\@startsection{subparagraph}{6}{\parindent}%
  {3.25ex \@plus1ex \@minus .2ex}%
  {-1em}%
  {\normalfont\normalsize\bfseries}}
\def\toclevel@subsubsubsection{4}
\def\toclevel@paragraph{5}
\def\toclevel@paragraph{6}
\def\l@subsubsubsection{\@dottedtocline{4}{7em}{4em}}
\def\l@paragraph{\@dottedtocline{5}{10em}{5em}}
\def\l@subparagraph{\@dottedtocline{6}{14em}{6em}}
\makeatother

\newcommand{\volume}{\mathop{\ooalign{\hfil$V$\hfil\cr\kern0.08em--\hfil\cr}}\nolimits}

\setcounter{secnumdepth}{4}
\setcounter{tocdepth}{4}
\begin{document}

\title{MATH 228b: HW2 ... 2bc, 3, 4}
\author{April Novak}

\maketitle

\section{} 

The solution of the following problem with linear Lagrange elements for \(u(0)=0\) and \(u'(1)=g\) is equivalent to a second order central difference approximation of the problem. 

\beq
\label{eq:1}
-u''(x)=f(x)
\eeq

The function space for the weight function \(v\) is in the space \(V_h\) such that \(v\) is \(C^0\) continuous over the entire domain (the zeroth derivative is the highest continuous derivative), where over each element \(K\), \(v\) is a linear function. \(v\) satisfies the homogeneous form of the Dirichlet boundary condition.

\beq
V_h=\{v\in C^0([0,1]): v\rvert_K\in P_1(K) \forall K\in T_h, v(0)=0\}
\eeq

The weighted residual form is obtained by multiplying Eq. \eqref{eq:1} by \(v\) and integrating over the domain. The first term on the left-hand side can be rewritten using the divergence rule:

\beqa
-\int_0^1 u''(x)v(x)dx=& \int_0^1 f(x)v(x)dx
\int_0^1 u'(x)v'(x)dx-\left\lbrack u'(x)v(x)\right\rbrack_0^1=& \int_0^1 f(x)v(x)dx
\eeqa

At this point, the weighted residual form has been reduced to the weak form. The weak form above contains no information about the boundary conditions. By applying the boundary conditions, the above reduces to:

\beq
\int_0^1 u'(x)v'(x)dx=\int_0^1 f(x)v(x)dx+gv(1)
\eeq

Using a linear Lagrange basis, defined on a mesh with element sizes \(h\), the elements are defined on \(e^1: [0, h]; e^2: [h, 2h]; e^3: [3h, 4h]; \cdots; e^n: [1-h, h]\), where \(n\) is the number of elements. The left and right coordinates of each element will be denoted as \(x_i\) and \(x_{i+1}\) for simplicity. Then, over an arbitrary element, the linear Lagrange shape functions \(\psi\) are:

\beqa
\psi_i(x)=& \frac{x_{i+1}-x}{x_{i+1}-x_i}=\frac{x_{i+1}-x}{h}\\
\psi_{i+1}(x) =& \frac{x-x_i}{x_{i+1}-x_i}=\frac{x-x_i}{h}\\
\eeqa

And their derivatives with respect to \(x\) are:

\beqa
\psi_i'(x)=&-\frac{1}{h}\\
\psi_{i+1}'(x) =&\frac{1}{h}\\
\eeqa

For equal-sized elements without edges on the Neumann boundary, the element stiffness matrix and element load vector are given as:

\beqa
A_{ij}^k=&\int_{x_i}^{x_{i+1}}\psi_i'(x)v_j'(x)dx\\
f_j^k=&\int_{x_i}^{x_{i+1}}f(x)v_j(x)dx\\
\eeqa

By inserting the shape functions given above, \(\textbf{A}^k\) becomes:

\beqa
\textbf{A}^k=& \begin{bmatrix}	\int_{x_i}^{x_{i+1}}\frac{-1}{h}\frac{-1}{h}dx & 
										\int_{x_i}^{x_{i+1}}\frac{-1}{h}\frac{1}{h}dx\\
										\int_{x_i}^{x_{i+1}}\frac{1}{h}\frac{-1}{h}dx &
										\int_{x_i}^{x_{i+1}}\frac{1}{h}\frac{1}{h}dx
					\end{bmatrix}\\
			   =& \begin{bmatrix}	1/h & 
										-1/h\\
										-1/h &
										1/h
					\end{bmatrix}
\eeqa

And the load vector becomes, with \(f=1\) chosen for simplicity:

\beqa
\textbf{f}^{\ k}=& \begin{bmatrix}	\int_{x_i}^{x_{i+1}}\frac{x_{i+1}-x}{h}dx \\
										\int_{x_i}^{x_{i+1}}\frac{x-x_i}{h}dx
					\end{bmatrix}\\
			   =& \begin{bmatrix}	\frac{(x_{i+1}-x_i)^2}{2h}\\
										\frac{(x_{i+1}-x_i)^2}{2h}
					\end{bmatrix}\\
				=& \begin{bmatrix}	h/2\\ 
										h/2
					\end{bmatrix}
\eeqa

Then, by the connectivity matrix relating how the nodes touch each other, the above matrices are ``tesselated'' along the diagonal of the global \textbf{A} and \textbf{f}. Suppose for the time being that \(u(1)=0\), i.e. both ends have homogeneous Dirichlet conditions. Then, the matrix system before application of the Dirichlet conditions is:

\beqa
\label{eq:3}
\begin{bmatrix}
1/h & -1/h & & & & &\\
-1/h & 2/h & -1/h & & & & \\
 & -1/h & 2/h & -1/h & & & \\
 & & -1/h & 2/h & -1/h & & \\
 & & & & & &\\
\end{bmatrix}
\begin{bmatrix}u_1\\u_2\\\vdots\\u_{n+1}\end{bmatrix} = 
\begin{bmatrix}
h/2\\ h\\h\\\vdots\\h\\h/2
\end{bmatrix}
\eeqa

Now, to apply the Neumann condition at \(x=1\), the load vector for the last element (the element spanning \(1-h\leq x\leq1\) must be modified:

\beqa
\textbf{f}^{\ e=n}=& \begin{bmatrix}	\int_{x_i}^{x_{i+1}}\frac{x_{i+1}-x}{h}dx \\
										\int_{x_i}^{x_{i+1}}\frac{x-x_i}{h}dx+g
					\end{bmatrix}\\
			   =& \begin{bmatrix}	\frac{(x_{i+1}-x_i)^2}{2h}\\
										\frac{(x_{i+1}-x_i)^2}{2h}+g
					\end{bmatrix}\\
				=& \begin{bmatrix}	h/2\\ 
										h/2+g
					\end{bmatrix}
\eeqa

Then, the last equation in Eq. \eqref{eq:3} must be modified. To strongly apply the homogeneous Dirichlet condition at \(x=0\), the first row and column of \textbf{A} is eliminated, and post-processing will assign the value \(u_1=0\). The full matrix system, with boundary conditions \(u(0)=0, u'(1)=g\), is, for a four-element system (purely for illustration):

\beqa
\label{eq:3}
\begin{bmatrix}
 2/h & -1/h & 0 & 0 \\
 -1/h & 2/h & -1/h & 0 \\
 0 & -1/h & 2/h & -1/h\\
 0 & 0 & -1/h & 1/h\\
\end{bmatrix}
\begin{bmatrix}u_2\\u_3\\u_4\\u_5\end{bmatrix} = 
\begin{bmatrix}
h\\h\\h\\h/2+g
\end{bmatrix}
\eeqa

Divide both sides by \(h\) for future insight into our choice of \(f(x)=1\):

\beqa
\label{eq:3}\frac{1}{h^2}
\begin{bmatrix}
 2 & -1 & 0 & 0 \\
 -1 & 2 & -1 & 0 \\
 0 & -1 & 2 & -1\\
 0 & 0 & -1 & 1\\
\end{bmatrix}
\begin{bmatrix}u_2\\u_3\\u_4\\u_5\end{bmatrix} = 
\begin{bmatrix}
1\\1\\1\\(h/2+g)/h
\end{bmatrix}
\eeqa

The central difference approximation to a second derivative is:

\beq
u''(x)\approx \frac{u_{i+1}-2u_i+u_{i-1}}{h^2}
\eeq

Looking at the equation for node 3, for instance, we can express this equation for \(i=3\) in terms of the values on either side (\(u_2=u_{i-1}\) and \(u_4=u_{i+1}\)):

\beq
\label{eq:4}
-\frac{1}{h^2}\left(u_{i-1}-2u_i+u_{i+1}\right)=1
\eeq

Eq. \eqref{eq:4} is identical to central difference approximation of the original equation in Eq. \eqref{eq:1} - the second derivative is approximated as a central difference, while the right-hand-side reflects the (arbitrary) choice of setting \(f(x)=1\). The above equation holds for all nodes on the interior. The enforcement of the Neumann condition can also be seen analogous to a finite difference approximation. The equation for the node on the Neumann boundary (node 5, where \(i=5\)), is:

\beq
\frac{1}{h}(-u_{i-1}+u_i)=\frac{h}{2}+g
\eeq

It is clear to see that the above represents a forward Euler method, since the slope at node \(i\) depends only on the solution value at \(i-1\). The \(h/2\) appears from the tessellation of the element load vector in \textbf{f}, and hence if that part is neglected for the time begin, then the above is exactly equivalent to stating that:

\beq
u'(1)\approx\frac{u_5-u_4}{h}=g
\eeq

So, this shows that, for linear Lagrange elements in 1-D, the finite element method is equivalent to a central difference method on the interior nodes with an Euler method enforcing Neumann conditions on the boundaries.

\section{}

This question will solve the following boundary value problem on \(x\in[0,1]\):

\beq
u''''(x)=f(x)=480x-120
\eeq

with boundary conditions \(u(0)=u'(0)=u(1)=u'(1)=0\). 

\subsection{}

The Galerkin formulation seeks a solution \(u_h\) in the function space \(V_h\), which is the same function space as the weight function \(v\):

\beq
V_h=\{v\in C^1([0,1]):v\rvert_K\in P_3(K) \forall K\in T_h, v(0)=v(1)=0=v'(0)=v'(1)\}
\eeq

The only requirement on the function space \(V_h\) is that it satisfy the homogeneous form of the essential boundary conditions. For a fourth-order governing equation, the essential boundary conditions are represented by the values of \(u\) and \(u'\) on the boundaries, while the natural boundary conditions in this case are the values of \(u'''\) and \(u''\) on the boundaries. In order for \(v\) to be able to satisfy the essential boundary conditions, \(v\) must be of sufficiently high order to specify \(v\) and \(v'\) on the boundaries (which for a 1-D element requires four degrees of freedom per element, neglecting continuity and boundary conditions requirements for the time being). Multiplying the governing equation by a weight function \(v(x)\) and integrating over the domain gives the weighted residual formulation of the governing equation:

\beq
\int_{0}^{1}u''''(x)v(x)dx=\int_{0}^{1}f(x)v(x)dx
\eeq

Integrating by parts two times:

\beqa
-\int_{0}^{1}u'''(x)v'(x)dx+\left\lbrack u'''(x)v(x)\right\rbrack_0^1=& \int_{0}^{1}f(x)v(x)dx\\
\int_{0}^{1}u''(x)v''(x)dx+\left\lbrack u'''(x)v(x)\right\rbrack_0^1-\left\lbrack u''(x)v'(x)\right\rbrack_0^1=& \int_{0}^{1}f(x)v(x)dx
\eeqa

Then, applying the boundary conditions:

\beq
\int_{0}^{1}u''(x)v''(x)dx=\int_{0}^{1}f(x)v(x)dx \quad\forall v\in V_h
\eeq

Now that differentiation has been modified to act equally (to the extent possible based on the governing equation) on \(u\) and \(v\), the above represents the weak form. The above equation holds for both the true solution \(u\) and the finite element solution \(u_h\), and hence the above could also be written as follows, which matches that shown in this homework assignment:

\beq
\int_{0}^{1}u_h''(x)v''(x)dx=\int_{0}^{1}f(x)v(x)dx \quad\forall v\in V_h
\eeq

\subsection{}

Over a particular element, the solution is specified as a summation of coefficients \(a\) multiplied by basis functions \(\varphi\), where \(N\) is the number of shape functions per element (which for linear Lagrange elements is equivalent to the number of nodes per element):

\beq
u_h^e(x)=\sum_{i=1}^{N}a_i\varphi_i(x)
\eeq

The above expression uses an \(e\) in the superscript of \(u_h\) to refer to the fact that \(u_h^e\) is the solution only over a particular element \(e\) - we must find the solution over every element. Because the space is defined to contain cubic polynomials over each element, since we require four degrees of freedom per element (not yet assuming continuity between elements), the shape functions for this problem must be functions that can specify both types of essential boundary conditions - the value and the first derivative on the two boundary nodes for each element. This requires four shape functions per element. Hermite shape functions allow specification of both the value and derivative at edge nodes. There are four shape functions per element. These shape functions are defined over an element defined on \([a, b]\) as:

\beqa
\label{eq:5}
H_1(a)=1, H_1(b)=0, \frac{dH_1(a)}{dx}=0, \frac{dH_1(b)}{dx}=0\\
H_2(a)=0, H_2(b)=1, \frac{dH_2(a)}{dx}=0, \frac{dH_2(b)}{dx}=0\\
H_3(a)=0, H_3(b)=0, \frac{dH_3(a)}{dx}=1, \frac{dH_3(b)}{dx}=0\\
H_4(a)=0, H_4(b)=0, \frac{dH_4(a)}{dx}=0, \frac{dH_4(b)}{dx}=1\\
\eeqa

where \(H\) is used to represent the fact that these shape functions \(\psi\) are Hermite shape functions. For each of these shape functions, because they are cubic, they are uniquely represented by four coefficients \(c, d, e, f\) for a polynomial of the form:

\beqa
H_1(x)=&c_1x^3+d_1x^2+e_1x+f_1\\
H_2(x)=&c_2x^3+d_2x^2+e_2x+f_2\\
H_3(x)=&c_3x^3+d_3x^2+e_3x+f_3\\
H_4(x)=&c_4x^3+d_4x^2+e_4x+f_4\\
\eeqa

\beqa
\frac{dH_1(x)}{dx}=&3c_1x^2+2d_1x+e_1\\
\frac{dH_2(x)}{dx}=&3c_2x^2+2d_2x+e_2\\
\frac{dH_3(x)}{dx}=&3c_3x^2+2d_3x+e_3\\
\frac{dH_4(x)}{dx}=&3c_4x^2+2d_4x+e_4\\
\eeqa

So, for each element in the triangulation, these 16 coefficients must be determined by solving the following linear system for \textit{each} of the functions \(H_i\):

\beq
\begin{bmatrix}
a^3 & a^2 & a & 1\\
b^3 & b^2 & b & 1\\
3a^2 & 2a & 1 & 0\\
3b^2 & 2b & 1 & 0
\end{bmatrix}
\begin{bmatrix}
c_1 \\d_1 \\e_1 \\f_1
\end{bmatrix}
=
\begin{bmatrix}
1 \\ 0 \\ 0 \\ 0
\end{bmatrix}
\quad\quad \textrm{for } H_1
\eeq

\beq
\begin{bmatrix}
a^3 & a^2 & a & 1\\
b^3 & b^2 & b & 1\\
3a^2 & 2a & 1 & 0\\
3b^2 & 2b & 1 & 0
\end{bmatrix}
\begin{bmatrix}
c_2 \\d_2 \\e_2 \\f_2
\end{bmatrix}
=
\begin{bmatrix}
0 \\ 1 \\ 0 \\ 0
\end{bmatrix}
\quad\quad \textrm{for } H_2
\eeq

\beq
\begin{bmatrix}
a^3 & a^2 & a & 1\\
b^3 & b^2 & b & 1\\
3a^2 & 2a & 1 & 0\\
3b^2 & 2b & 1 & 0
\end{bmatrix}
\begin{bmatrix}
c_3 \\d_3 \\e_3 \\f_3
\end{bmatrix}
=
\begin{bmatrix}
0 \\ 0 \\ 1 \\ 0
\end{bmatrix}
\quad\quad \textrm{for } H_3
\eeq

\beq
\begin{bmatrix}
a^3 & a^2 & a & 1\\
b^3 & b^2 & b & 1\\
3a^2 & 2a & 1 & 0\\
3b^2 & 2b & 1 & 0
\end{bmatrix}
\begin{bmatrix}
c_4 \\d_4 \\e_4 \\f_4
\end{bmatrix}
=
\begin{bmatrix}
0 \\ 0 \\ 0 \\ 4
\end{bmatrix}
\quad\quad \textrm{for } H_4
\eeq

Then, this process is repeated for all the elements in the triangulation, where \(a\) and \(b\) will change corresponding to the endpoints of the domain. This system of equations is solved in Python, where the code is provided in the Appendix. This solution, for the two-element triangulation given, is, for element 1:

\beqa
H_1(x)=&16x^3-12x^2+1\\
H_2(x)=&-16x^3+12x^2\\
H_3(x)=&4x^3-4x^2+x\\
H_4(x)=&4x^3-2x^2\\
\eeqa

And for element 2:

\beqa
H_1(x)=&16x^3-36x^2+24x-4\\
H_2(x)=&-16x^3+36x^2-24x+5\\
H_3(x)=&4x^3-10x^2+8x-2\\
H_4(x)=&4x^3-8x^2+5x-1\\
\eeqa

Fig. \ref{fig:1} shows the shape functions over the two-element domain.

\begin{figure}[H]
\centering
\includegraphics[width=0.75\textwidth]{q2_Hermite_functions.png}
\caption{Hermite shape functions over the two-element domain.}
\label{fig:1}
\end{figure}

\end{document}

