\documentclass[10pt]{article}
\usepackage[letterpaper]{geometry}
\geometry{verbose,tmargin=1in,bmargin=1in,lmargin=1in,rmargin=1in}
\usepackage{setspace}
\usepackage{ragged2e}
\usepackage{color}
\usepackage{titlesec}
\usepackage{graphicx}
\usepackage{float}
\usepackage{mathtools}
\usepackage{amsmath}
\usepackage[font=small,labelfont=bf,labelsep=period]{caption}
\usepackage[english]{babel}
\usepackage{indentfirst}
\usepackage{array}
\usepackage{makecell}
\usepackage[usenames,dvipsnames]{xcolor}
\usepackage{multirow}
\usepackage{tabularx}
\usepackage{arydshln}
\usepackage{caption}
\usepackage{subcaption}
\usepackage{xfrac}
\usepackage{etoolbox}
\usepackage{cite}
\usepackage{url}
\usepackage{dcolumn}
\usepackage{hyperref}
\usepackage{courier}
\usepackage{esvect}
\usepackage{commath}
\usepackage{verbatim} % for block comments
\usepackage{enumitem}
\usepackage{hyperref} % for clickable table of contents
\usepackage{braket}
\usepackage{titlesec}
\usepackage{booktabs}
\usepackage{gensymb}
\usepackage{listings}
\usepackage{cancel}
\usepackage[mathscr]{euscript}
\lstset{
    frame=single,
    breaklines=true,
    postbreak=\raisebox{0ex}[0ex][0ex]{\ensuremath{\color{red}\hookrightarrow\space}}
}

% for circled numbers
\usepackage{tikz}
\newcommand*\circled[1]{\tikz[baseline=(char.base)]{
            \node[shape=circle,draw,inner sep=2pt] (char) {#1};}}

\newcommand{\beq}{\begin{equation}}
\newcommand{\eeq}{\end{equation}}
\newcommand{\beqa}{\begin{equation}\begin{aligned}}
\newcommand{\eeqa}{\end{aligned}\end{equation}}

\titleclass{\subsubsubsection}{straight}[\subsection]

% define new command for triple sub sections
\newcounter{subsubsubsection}[subsubsection]
\renewcommand\thesubsubsubsection{\thesubsubsection.\arabic{subsubsubsection}}
\renewcommand\theparagraph{\thesubsubsubsection.\arabic{paragraph}} % optional; useful if paragraphs are to be numbered

\titleformat{\subsubsubsection}
  {\normalfont\normalsize\bfseries}{\thesubsubsubsection}{1em}{}
\titlespacing*{\subsubsubsection}
{0pt}{3.25ex plus 1ex minus .2ex}{1.5ex plus .2ex}

\makeatletter
\renewcommand\paragraph{\@startsection{paragraph}{5}{\z@}%
  {3.25ex \@plus1ex \@minus.2ex}%
  {-1em}%
  {\normalfont\normalsize\bfseries}}
\renewcommand\subparagraph{\@startsection{subparagraph}{6}{\parindent}%
  {3.25ex \@plus1ex \@minus .2ex}%
  {-1em}%
  {\normalfont\normalsize\bfseries}}
\def\toclevel@subsubsubsection{4}
\def\toclevel@paragraph{5}
\def\toclevel@paragraph{6}
\def\l@subsubsubsection{\@dottedtocline{4}{7em}{4em}}
\def\l@paragraph{\@dottedtocline{5}{10em}{5em}}
\def\l@subparagraph{\@dottedtocline{6}{14em}{6em}}
\makeatother

\newcommand{\volume}{\mathop{\ooalign{\hfil$V$\hfil\cr\kern0.08em--\hfil\cr}}\nolimits}

\setcounter{secnumdepth}{4}
\setcounter{tocdepth}{4}
\begin{document}

\title{MATH 228b: HW3 ... 1c, 2bcd, 3abc, 4bc}
\author{April Novak}

\maketitle

\section{}

This problem solves the following differential equation:

\beq
\label{eq:prob}
-\nabla^2 u-k^2u=0
\eeq

With boundary conditions:

\beqa
\label{eq:BCs}
\hat{n}\cdot\nabla u=&0 & \textrm{ on } \Gamma_{wall}\\
\hat{n}\cdot\nabla u=&-iku & \textrm{ on } \Gamma_{out}\\
\hat{n}\cdot\nabla u=&2ik-iku & \textrm{ on } \Gamma_{in}\\
\eeqa

where the domain boundary is divided into three sections, where \(\Gamma=\Gamma_{wall}\cup\Gamma_{out}\cup\Gamma_{in}\). 

\subsection{(a)}

The weighted residual form is obtained by multiplying the governing equation by a weight function \(v\) and integrating over the domain, applying integration by parts when possible:

\beqa
-\int_{\Omega}v\nabla^2 u-\int_{\Omega}k^2uv=& 0\\
\int_{\Omega}\nabla u\cdot\nabla v-\int_{\Gamma}\hat{n}v\cdot\nabla u-\int_{\Omega}k^2uv=&0\\
\eeqa

Applying the boundary conditions:

\beq
\int_{\Omega}\nabla u\cdot\nabla v+\int_{\Gamma_{out}}viku-\int_{\Gamma_{in}}v(2ik-iku)-\int_{\Omega}k^2uv=0
\eeq

\(u\) is expanded in linear continuous polynomials over a series of elements \(K\) defined by a triangulation \(T_h\) such that:

\beq
u_h=\{u\in C^0(\Omega):u\rvert_K\in P_1(K)\forall K\in T_h, \textrm{for } \hat{n}\cdot\nabla u=0 \textrm{ on } \Gamma_{wall}, \hat{n}\cdot\nabla u=-iku \textrm{ on } \Gamma_{out}, \hat{n}\cdot\nabla u=2ik-iku \textrm{ on } \Gamma_{in}\}
\eeq

For a Galerkin approximation, \(v\) is chosen from the space \(V_h\), which is the same space as \(u_h\) except that \(V_h\) satisfy the homogeneous form of the essential boundary conditions (of which there are none in this problem):

\beq
V_h=\{v\in C^0(\Omega):v\rvert_K\in P_1(K)\forall K\in T_h\}
\eeq

The Galerkin finite element problem is therefore:

\beq
\int_{\Omega}\nabla u_h\cdot\nabla v_h+\int_{\Gamma_{out}}v_hiku_h+\int_{\Gamma_{in}}v_hiku_h-\int_{\Omega}k^2u_hv_h=\int_{\Gamma_{in}}v_h2ik
\eeq

\subsection{(b)}

\(u_h\) is represented as an expansion of basis functions described by the spaces given above:

\beq
u_h=\sum_{i=1}^{N}a_i\phi_i
\eeq

where \(N\) are the number of basis functions over the entire domain. Alternatively, this can be expressed in terms of the solution over each finite element:

\beq
u_h^e=\sum_{i=1}^{n_{en}}a_i\phi_i
\eeq

where \(n_{en}\) are the number of nodes per element and the \(e\) superscript indicates that the solution is only over a single element, with piecewise continuity between elements. Inserting these shape functions into the weak form above, where \(v_h\) is likewise expanded in the same set of basis functions as \(u_h\), but with expansion coefficients \(b\):

\beqa
\int_{\Omega}\nabla \left(\sum_{i=1}^{N}a_i\phi_i\right)\cdot\nabla \left(\sum_{j=1}^{N}b_j\phi_j\right)+\int_{\Gamma_{out}}\left(\sum_{j=1}^{N}b_j\phi_j\right)ik\left(\sum_{i=1}^{N}a_i\phi_i\right)+\int_{\Gamma_{in}}\left(\sum_{j=1}^{N}b_j\phi_j\right)ik\left(\sum_{i=1}^{N}a_i\phi_i\right)\\
-\int_{\Omega}k^2\left(\sum_{i=1}^{N}a_i\phi_i\right)\left(\sum_{j=1}^{N}b_j\phi_j\right)=\int_{\Gamma_{in}}\left(\sum_{j=1}^{N}b_j\phi_j\right)2ik
\eeqa

A simpler way of representing this weak form is to recognize that \(u_h\) and \(v_h\) can be represented as:

\beqa
u_h=&\textbf{N}\textbf{u}=\left\lbrack\phi_1, \phi_2, \cdots, \phi_N\right\rbrack\left\lbrack a_1, a_2, \cdots, a_N\right\rbrack^T\\
v_h=&\textbf{N}\textbf{v}=\left\lbrack\phi_1, \phi_2, \cdots, \phi_N\right\rbrack\left\lbrack b_1, b_2, \cdots, b_N\right\rbrack^T\\
\eeqa

The gradient of \(u_h\) is:

\beq
\nabla u_h=\textbf{B}\textbf{u}=\begin{bmatrix} 
\frac{\partial \phi_1}{\partial x} & \frac{\partial \phi_2}{\partial x} & \cdots & \frac{\partial \phi_N}{\partial x}\\
\frac{\partial \phi_1}{\partial y} & \frac{\partial \phi_2}{\partial y} & \cdots & \frac{\partial \phi_N}{\partial y}\\
\end{bmatrix}\textbf{u}
\eeq

The weak form can therefore equivalently be expressed as:

\beqa
\int_{\Omega}(\textbf{B}\textbf{u})\cdot(\textbf{B}\textbf{v})+\int_{\Gamma_{out}}(\textbf{N}\textbf{v})^Tik(\textbf{N}\textbf{u})+\int_{\Gamma_{in}}(\textbf{N}\textbf{v})^Tik(\textbf{N}\textbf{u})-\int_{\Omega}k^2(\textbf{N}\textbf{v})^T(\textbf{N}\textbf{u})=&\int_{\Gamma_{in}}(\textbf{N}\textbf{v})2ik\\
\int_{\Omega}(\textbf{B}\textbf{v})^T(\textbf{B}\textbf{u})+\int_{\Gamma_{out}}(\textbf{N}\textbf{v})^Tik(\textbf{N}\textbf{u})+\int_{\Gamma_{in}}(\textbf{N}\textbf{v})^Tik(\textbf{N}\textbf{u})-\int_{\Omega}k^2(\textbf{N}\textbf{v})^T(\textbf{N}\textbf{u})=&\int_{\Gamma_{in}}(\textbf{N}\textbf{v})^T2ik\\
\eeqa

Because \(\textbf{v}^T\) appears in every term, the above could be rearranged such that \(\textbf{v}^T\) multiplies an integrand, which equals zero, which would lead to the conclusion that the integrand itself must equal zero. In other words, \(\textbf{v}^T\) can essentially be canceled from every term.

\beqa
\int_{\Omega}\textbf{B}^T(\textbf{B}\textbf{u})+\int_{\Gamma_{out}}\textbf{N}^Tik(\textbf{N}\textbf{u})+\int_{\Gamma_{in}}\textbf{N}^Tik(\textbf{N}\textbf{u})-\int_{\Omega}k^2\textbf{N}^T(\textbf{N}\textbf{u})=&\int_{\Gamma_{in}}\textbf{N}^T2ik\\
\eeqa

To obtain a matrix equation in the form of that in the assignment, define:

\beqa
\textbf{K}=&\int_{\Omega}\textbf{B}^T\textbf{B}d\Omega\\
\textbf{M}=&\int_{\Omega}\textbf{N}^T\textbf{N}d\Omega\\
\textbf{B}_{in}=&\int_{\Gamma_{in}}\textbf{N}^T\textbf{N}d\Gamma\\
\textbf{B}_{out}=&\int_{\Gamma_{out}}\textbf{N}^T\textbf{N}d\Gamma\\
\textbf{b}_{in}=&\int_{\Gamma_{in}}\textbf{N}^Td\Gamma\\
\eeqa

Then, the overall matrix system is:

\beq
\left(\textbf{K}-k^2\textbf{M}+ik(\textbf{B}_{in}+\textbf{B}_{out})\right)\textbf{u}=2ik\textbf{b}_{in}
\eeq

Refer to the definitions of \textbf{N} and \textbf{B} to see these definitions in terms of the basis functions.

\subsection{(c)}

The transmitted intensity, when computed as a function of the finite element solution, is:

\beq
H(u_h)=\int_{\Gamma_{out}}|\textbf{N}\textbf{u}|^2
\eeq

where \(|(.)|\) is the complex absolute value of \((.)\). 

\section{}

This problem solves the problem stated in Eq. \eqref{eq:prob} for boundary conditions given in Eq. \eqref{eq:BCs} for the rectangular domain given in the assignment.

\subsection{(a)}

The analytical solution to Eq. \eqref{eq:prob} with boundary conditions Eq. \eqref{eq:BCs} is only a function of \(x\), since the insulation boundary conditions along the top and bottom edges of the domain are symmetric. Hence, the PDE reduces to an ODE:

\beq
\frac{d^2u}{dx^2}+k^2u=0
\eeq

This common ODE has the following general solution:

\beqa
u(x)=&C_1sin(kx)+C_2cos(kx)\\
u(x)=&C_1\frac{1}{2i}\left(e^{ikx}-e^{-ikx}\right)+\frac{C_2}{2}\left(e^{ikx}+e^{-ikx}\right)\\
\eeqa

Apply the boundary condition on \(\Gamma_{in}\), where \(\hat{n}=\lbrack -1, 0\rbrack\) and \(x=0\):

\beqa
\lbrack -1, 0\rbrack\cdot(C_1kcos(kx)-C_2ksin(kx))=&2ik-ik(C_1sin(kx)+C_2cos(kx))\\
C_1=&-\frac{2ik-ikC_2}{k}\\
\eeqa

Applying the boundary condition on \(\Gamma_{out}\), where \(\hat{n}=\lbrack 1, 0\rbrack\) and \(x=5\):

\beqa
\lbrack 1, 0\rbrack\cdot(C_1kcos(5k)-C_2ksin(5k))=&-ik(C_1sin(5k)+C_2cos(5k))\\
C_1kcos(5k)-C_2ksin(5k)=&-ik(C_1sin(5k)+C_2cos(5k))\\
\eeqa

\beqa
u(x)=&-\frac{2-C_2}{2}\left(e^{ikx}-e^{-ikx}\right)+\frac{C_2}{2}\left(e^{ikx}+e^{-ikx}\right)\\
\eeqa

Applying the boundary condition on \(\Gamma_{out}\), where \(\hat{n}=\lbrack 1, 0\rbrack\) and \(x=5\):

\beqa
-\frac{2-C_2}{2}\left(e^{ikx}+e^{-ikx}\right)+\frac{C_2}{2}\left(e^{ikx}-e^{-ikx}\right)=&\frac{2-C_2}{2}\left(e^{ikx}-e^{-ikx}\right)-\frac{C_2}{2}\left(e^{ikx}+e^{-ikx}\right)\\
\frac{C_2}{2}\left\lbrack\left(e^{ikx}-e^{-ikx}+\left(e^{ikx}+e^{-ikx}\right)\right)\right\rbrack=&\frac{2-C_2}{2}\left\lbrack\left(e^{ikx}-e^{-ikx}\right)+\left(e^{ikx}+e^{-ikx}\right)\right\rbrack\\
\frac{C_2}{2}\left\lbrack e^{ikx}+e^{ikx}\right\rbrack=&\frac{2-C_2}{2}\left\lbrack e^{ikx}+e^{ikx}\right\rbrack\\
\frac{C_2}{2}=&\frac{2-C_2}{2}\\
C_2=&1\\
\eeqa

Then, \(C_1=-i\). Plugging these coefficients into the general solution gives the general solution for these boundaries.

\beqa
u(x)=&cos(kx)-isin(kx)\\
u(x)=&e^{-ikx}\\
\eeqa

\subsection{(b)}

\section{}

\section{}

This problem extends the {\tt fempoi.m} solver developed in the previous homework for quadratic triangular elements. All triangle are assumed to have straight edges. 

\subsection{(a)}

A mesh developed for linear basis functions can easily be extended to a mesh for quadratic basis functions by simply adding three additional nodes per triangle, one at the midpoint of each side. The additional points are added to the original mesh by looping over the triangles and adding the midpoints of every side, and the calling {\tt unique} to remove (roughly) half of these points which are shared by two triangles. Then, a loop is created that loops over all of the original triangles and re-computes these midpoints and then searches through the (original + new) list of points to determine the row number of that point. Then, the new triangulation is defined such that the corners of the original triangle fill the 1, 3, and 5 entries of the new triangulation, with the 2, 4, and 6 entries given as the row number of the midpoint along the corresponding edge.

To determine which of the original nodes (in {\tt p}) are on the boundary, {\tt boundary\_nodes} is used. Then, to determine which of the new nodes are on the boundary, if two nodes in the same triangle are both on the boundary, then the midpoint node on the edge between those two corner points is in most situations also on the boundary. However, in some cases, when all three corner nodes of a triangle are on the boundary, then at least one of the midpoint nodes is actually not in the boundary. So, a loop over all of the rows and columns in the triangulation is used to determine if a point appears more than once in the triangulation - if so, it is not actually on the boundary. In this way, the boundary nodes, consisting of the original boundary nodes and the new boundary nodes, is created. The code developed for this section is shown in the Appendix.

\section{Appendix}
\subsection{Question 4}
\subsubsection{Part (a) - {\tt p2mesh.m}}
\lstinputlisting[language=Matlab]{p2mesh.m}

\end{document}
