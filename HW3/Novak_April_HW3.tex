\documentclass[10pt]{article}
\usepackage[letterpaper]{geometry}
\geometry{verbose,tmargin=1in,bmargin=1in,lmargin=1in,rmargin=1in}
\usepackage{setspace}
\usepackage{ragged2e}
\usepackage{color}
\usepackage{titlesec}
\usepackage{graphicx}
\usepackage{float}
\usepackage{mathtools}
\usepackage{amsmath}
\usepackage[font=small,labelfont=bf,labelsep=period]{caption}
\usepackage[english]{babel}
\usepackage{indentfirst}
\usepackage{array}
\usepackage{makecell}
\usepackage[usenames,dvipsnames]{xcolor}
\usepackage{multirow}
\usepackage{tabularx}
\usepackage{arydshln}
\usepackage{caption}
\usepackage{subcaption}
\usepackage{xfrac}
\usepackage{etoolbox}
\usepackage{cite}
\usepackage{url}
\usepackage{dcolumn}
\usepackage{hyperref}
\usepackage{courier}
\usepackage{url}
\usepackage{esvect}
\usepackage{commath}
\usepackage{verbatim} % for block comments
\usepackage{enumitem}
\usepackage{hyperref} % for clickable table of contents
\usepackage{braket}
\usepackage{titlesec}
\usepackage{booktabs}
\usepackage{gensymb}
\usepackage{longtable}
\usepackage{listings}
\usepackage{cancel}
\usepackage{tcolorbox}
\usepackage[mathscr]{euscript}
\lstset{
    frame=single,
    breaklines=true,
    postbreak=\raisebox{0ex}[0ex][0ex]{\ensuremath{\color{red}\hookrightarrow\space}}
}

% for circled numbers
\usepackage{tikz}
\newcommand*\circled[1]{\tikz[baseline=(char.base)]{
            \node[shape=circle,draw,inner sep=2pt] (char) {#1};}}

\newcommand{\beq}{\begin{equation}}
\newcommand{\eeq}{\end{equation}}
\newcommand{\beqa}{\begin{equation}\begin{aligned}}
\newcommand{\eeqa}{\end{aligned}\end{equation}}

\titleclass{\subsubsubsection}{straight}[\subsection]

% define new command for triple sub sections
\newcounter{subsubsubsection}[subsubsection]
\renewcommand\thesubsubsubsection{\thesubsubsection.\arabic{subsubsubsection}}
\renewcommand\theparagraph{\thesubsubsubsection.\arabic{paragraph}} % optional; useful if paragraphs are to be numbered

\titleformat{\subsubsubsection}
  {\normalfont\normalsize\bfseries}{\thesubsubsubsection}{1em}{}
\titlespacing*{\subsubsubsection}
{0pt}{3.25ex plus 1ex minus .2ex}{1.5ex plus .2ex}

\makeatletter
\renewcommand\paragraph{\@startsection{paragraph}{5}{\z@}%
  {3.25ex \@plus1ex \@minus.2ex}%
  {-1em}%
  {\normalfont\normalsize\bfseries}}
\renewcommand\subparagraph{\@startsection{subparagraph}{6}{\parindent}%
  {3.25ex \@plus1ex \@minus .2ex}%
  {-1em}%
  {\normalfont\normalsize\bfseries}}
\def\toclevel@subsubsubsection{4}
\def\toclevel@paragraph{5}
\def\toclevel@paragraph{6}
\def\l@subsubsubsection{\@dottedtocline{4}{7em}{4em}}
\def\l@paragraph{\@dottedtocline{5}{10em}{5em}}
\def\l@subparagraph{\@dottedtocline{6}{14em}{6em}}
\makeatother

\newcommand{\volume}{\mathop{\ooalign{\hfil$V$\hfil\cr\kern0.08em--\hfil\cr}}\nolimits}

\setcounter{secnumdepth}{4}
\setcounter{tocdepth}{4}
\begin{document}

\title{MATH 228b: HW3 ... 1bc, 2, 3, 4}
\author{April Novak}

\maketitle

\section{}

This problem solves the following differential equation:

\beq
-\nabla^2 u-k^2u=0
\eeq

With boundary conditions:

\beqa
\label{eq:BCs}
\hat{n}\cdot\nabla u=&0 & \textrm{ on } \Gamma_{wall}\\
\hat{n}\cdot\nabla u=&-iku & \textrm{ on } \Gamma_{out}\\
\hat{n}\cdot\nabla u=&2ik-iku & \textrm{ on } \Gamma_{in}\\
\eeqa

where the domain boundary is divided into three sections, where \(\Gamma=\Gamma_{wall}\cup\Gamma_{out}\cup\Gamma_{in}\). 

\subsection{(a)}

The weighted residual form is obtained by multiplying the governing equation by a weight function \(v\) and integrating over the domain, applying integration by parts when possible:

\beqa
-\int_{\Omega}v\nabla^2 u-\int_{\Omega}k^2uv=& 0\\
\int_{\Omega}\nabla u\cdot\nabla v-\int_{\Gamma}\hat{n}v\cdot\nabla u-\int_{\Omega}k^2uv=&0\\
\eeqa

Applying the boundary conditions:

\beq
\int_{\Omega}\nabla u\cdot\nabla v+\int_{\Gamma_{out}}viku-\int_{\Gamma_{in}}v(2ik-iku)-\int_{\Omega}k^2uv=0
\eeq

\(u\) is expanded in linear continuous polynomials over a series of elements \(K\) defined by a triangulation \(T_h\) such that:

\beq
u_h=\{u\in C^0(\Omega):u\rvert_K\in P_1(K)\forall K\in T_h, \textrm{for } \hat{n}\cdot\nabla u=0 \textrm{ on } \Gamma_{wall}, \hat{n}\cdot\nabla u=-iku \textrm{ on } \Gamma_{out}, \hat{n}\cdot\nabla u=2ik-iku \textrm{ on } \Gamma_{in}\}
\eeq

For a Galerkin approximation, \(v\) is chosen from the space \(V_h\), which is the same space as \(u_h\) except that \(V_h\) satisfy the homogeneous form of the essential boundary conditions (of which there are none in this problem):

\beq
V_h=\{v\in C^0(\Omega):v\rvert_K\in P_1(K)\forall K\in T_h\}
\eeq

The Galerkin finite element problem is therefore:

\beq
\int_{\Omega}\nabla u_h\cdot\nabla v_h+\int_{\Gamma_{out}}v_hiku_h+\int_{\Gamma_{in}}v_hiku_h-\int_{\Omega}k^2u_hv_h=\int_{\Gamma_{in}}v_h2ik
\eeq

\end{document}
