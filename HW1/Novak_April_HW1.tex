\documentclass[10pt]{article}
\usepackage[letterpaper]{geometry}
\geometry{verbose,tmargin=1in,bmargin=1in,lmargin=1in,rmargin=1in}
\usepackage{setspace}
\usepackage{ragged2e}
\usepackage{color}
\usepackage{titlesec}
\usepackage{graphicx}
\usepackage{float}
\usepackage{mathtools}
\usepackage{amsmath}
\usepackage[font=small,labelfont=bf,labelsep=period]{caption}
\usepackage[english]{babel}
\usepackage{indentfirst}
\usepackage{array}
\usepackage{makecell}
\usepackage[usenames,dvipsnames]{xcolor}
\usepackage{multirow}
\usepackage{tabularx}
\usepackage{arydshln}
\usepackage{caption}
\usepackage{subcaption}
\usepackage{xfrac}
\usepackage{etoolbox}
\usepackage{cite}
\usepackage{url}
\usepackage{dcolumn}
\usepackage{hyperref}
\usepackage{courier}
\usepackage{url}
\usepackage{esvect}
\usepackage{commath}
\usepackage{verbatim} % for block comments
\usepackage{enumitem}
\usepackage{hyperref} % for clickable table of contents
\usepackage{braket}
\usepackage{titlesec}
\usepackage{booktabs}
\usepackage{gensymb}
\usepackage{longtable}
\usepackage{listings}
\usepackage{cancel}
\usepackage{tcolorbox}
\usepackage[mathscr]{euscript}
\lstset{
    frame=single,
    breaklines=true,
    postbreak=\raisebox{0ex}[0ex][0ex]{\ensuremath{\color{red}\hookrightarrow\space}}
}

% for circled numbers
\usepackage{tikz}
\newcommand*\circled[1]{\tikz[baseline=(char.base)]{
            \node[shape=circle,draw,inner sep=2pt] (char) {#1};}}


\titleclass{\subsubsubsection}{straight}[\subsection]

% define new command for triple sub sections
\newcounter{subsubsubsection}[subsubsection]
\renewcommand\thesubsubsubsection{\thesubsubsection.\arabic{subsubsubsection}}
\renewcommand\theparagraph{\thesubsubsubsection.\arabic{paragraph}} % optional; useful if paragraphs are to be numbered

\titleformat{\subsubsubsection}
  {\normalfont\normalsize\bfseries}{\thesubsubsubsection}{1em}{}
\titlespacing*{\subsubsubsection}
{0pt}{3.25ex plus 1ex minus .2ex}{1.5ex plus .2ex}

\makeatletter
\renewcommand\paragraph{\@startsection{paragraph}{5}{\z@}%
  {3.25ex \@plus1ex \@minus.2ex}%
  {-1em}%
  {\normalfont\normalsize\bfseries}}
\renewcommand\subparagraph{\@startsection{subparagraph}{6}{\parindent}%
  {3.25ex \@plus1ex \@minus .2ex}%
  {-1em}%
  {\normalfont\normalsize\bfseries}}
\def\toclevel@subsubsubsection{4}
\def\toclevel@paragraph{5}
\def\toclevel@paragraph{6}
\def\l@subsubsubsection{\@dottedtocline{4}{7em}{4em}}
\def\l@paragraph{\@dottedtocline{5}{10em}{5em}}
\def\l@subparagraph{\@dottedtocline{6}{14em}{6em}}
\makeatother

\newcommand{\volume}{\mathop{\ooalign{\hfil$V$\hfil\cr\kern0.08em--\hfil\cr}}\nolimits}

\setcounter{secnumdepth}{4}
\setcounter{tocdepth}{4}
\begin{document}

\title{MATH 228b: HW1 ..... 1, 2, 3, 4}
\author{April Novak}

\maketitle

\section{}

Transfinite Interpolation (TFI) is used to map from a master domain (defined in 2-D over \(0\leq\xi\leq1\) and \(0\leq\eta\leq1\)) to the physical domain by expressing the boundary of the physical domain in terms of \(\xi,\eta\) coordinates. This then constrains the master grid to lie within the physical domain boundaries. TFI does not guarantee a one-to-one mapping or orthogonality of the mesh, however, but is useful for directly controlling the grid spacing in a computationally efficient and easy manner. The mapping from the master domain to the physical domain is given by \(\textbf{X}(\xi,\eta)\):

\begin{equation}
\textbf{X}(\xi,\eta)=\left\lbrack x(\xi,\eta), y(\xi,\eta)\right\rbrack^T
\end{equation}

where \(x(\xi,\eta)\) and \(y(\xi,\eta)\) are the mappings from the individual coordinates in the master domain (\(\xi,\eta\)) to those in the physical domain (\(x,y\)). For \textit{linear} TFI, we construct 1-D, \textit{linear} interpolants in the \(x\) and \(y\) directions:

\begin{equation}
\begin{aligned}
\textbf{U}(\xi,\eta)=& (1-\xi_i)\textbf{X}(0,\eta_j)+\xi_i\textbf{X}(1,\eta_j)\\
\textbf{V}(\xi,\eta)=& (1-\eta_j)\textbf{X}(\xi_i,0)+\eta_j\textbf{X}(\xi_i,1)\\
\end{aligned}
\end{equation}

where \(\textbf{U}\) is the 1-D interpolant in the \(x\)-direction and \(\textbf{V}\) the 1-D interpolant in the \(y\)-direction. For a 2-D domain, \(x\) and \(y\) are functions of \(\xi\) and \(\eta\). The 2-D interpolant is constructed from the 1-D interpolants above by performing a Boolean sum.

\begin{equation}
\begin{aligned}
\textbf{X}(\xi,\eta)=& (1-\xi_i)\textbf{X}(0,\eta_j)+\xi_i\textbf{X}(1,\eta_j) + (1-\eta_j)\textbf{X}(\xi_i,0)+\eta_j\textbf{X}(\xi_i,1) - \\
& \left\lbrack(1-\xi_i)(1-\eta_j)\textbf{X}(0,0)+(1-\xi_i)\eta_j\textbf{X}(0,1)+\xi_i(1-\eta_j)\textbf{X}(1,0)+\xi_i\eta_j\textbf{X}(1,1)\right\rbrack
\end{aligned}
\end{equation}

The \(\textbf{X}\) that appear on the right-hand side of the equation above are assumed to be known, and is how the physical domain boundary enters the meshing algorithm. For a straight-sided domain, with corner coordinates \((x_1,y_1), (x_2,y_2), (x_3,y_3), (x_4, y_4)\), where the first coordinate pair is in the lower left corner, the interpolation along the sides is given by:

\begin{equation}
\begin{aligned}
\textbf{X}(0,\eta_j)=\begin{bmatrix}(1-\eta_j)x_1+\eta x_4\\ (1-\eta_j)y_0+\eta y_4\end{bmatrix}\\
\textbf{X}(1,\eta_j)=\begin{bmatrix}(1-\eta_j)x_2+\eta x_3\\ (1-\eta_j)y_2+\eta y_3\end{bmatrix}\\
\textbf{X}(\xi_i,0)=\begin{bmatrix}(1-\xi_i)x_1+\xi_i x_2\\ (1-\xi_i)y_1+\xi_i y_2\end{bmatrix}\\
\textbf{X}(\xi_i,1)=\begin{bmatrix}(1-\xi_i)x_4+\xi_i x_3\\ (1-\xi_i)y_4+\xi_i y_3\end{bmatrix}\\
\end{aligned}
\end{equation}

where a 2-D domain is used for simplicity - the results would equally extend to 3-D. Then, inserting these into the Boolean sum above gives:

\begin{equation}
\begin{aligned}
x(\xi,\eta)=& (1-\xi_i)\left((1-\eta_j)x_1+\eta_j x_4\right)+\xi_i\left((1-\eta_j)x_2+\eta x_3\right) + \\
&(1-\eta_j)\left((1-\xi_i)x_1+\xi_i x_2\right)+\eta_j\left((1-\xi_i)x_4+\xi_i x_3\right) - \\
& \left\lbrack(1-\xi_i)(1-\eta_j)x_1+(1-\xi_i)\eta_jx_4+\xi_i(1-\eta_j)x_2+\xi_i\eta_jx_3\right\rbrack\\
=& \cancel{(1-\xi_i)(1-\eta_j)x_1}+\cancel{(1-\xi_i)\eta_j x_4}+\cancel{\xi_i(1-\eta_j)x_2}+\cancel{\eta_j x_3\xi_i} +\\
&(1-\eta_j)(1-\xi_i)x_1+\xi_i x_2(1-\eta_j)+\eta_j(1-\xi_i)x_4+\xi_i x_3\eta_j-\\
& \left\lbrack\cancel{(1-\xi_i)(1-\eta_j)x_1}+\cancel{(1-\xi_i)\eta_jx_4}+\cancel{\xi_i(1-\eta_j)x_2}+\cancel{\xi_i\eta_jx_3}\right\rbrack\\
=& (1-\eta_j)(1-\xi_i)x_1+\xi_i x_2(1-\eta_j)+\eta_j(1-\xi_i)x_4+x_3\eta_j\xi_i\\
\end{aligned}
\end{equation}

From the last line above, it is seen that, for the transformation from \(\xi\) to \(x\), the linear TFI is equivalent to bilinear interpolation between the four corner points. Similarly, bilinear interpolation between the \(y\)-coordinates of the four corner points is also equivalent to a bilinear interpolation (work follows the same steps shown above).

\begin{equation}
\begin{aligned}
y(\xi,\eta)= (1-\eta_j)(1-\xi_i)y_1+\xi_i y_2(1-\eta_j)+\eta_j(1-\xi_i)y_4+ y_3\eta_j\xi_i\\
\end{aligned}
\end{equation}

\section{}

\section{}

\section{}

\end{document}