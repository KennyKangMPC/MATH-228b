\documentclass[10pt]{article}
\usepackage[letterpaper]{geometry}
\geometry{verbose,tmargin=1in,bmargin=1in,lmargin=1in,rmargin=1in}
\usepackage{setspace}
\usepackage{ragged2e}
\usepackage{color}
\usepackage{titlesec}
\usepackage{graphicx}
\usepackage{float}
\usepackage{mathtools}
\usepackage{amsmath}
\usepackage[font=small,labelfont=bf,labelsep=period]{caption}
\usepackage[english]{babel}
\usepackage{indentfirst}
\usepackage{array}
\usepackage{makecell}
\usepackage[usenames,dvipsnames]{xcolor}
\usepackage{multirow}
\usepackage{tabularx}
\usepackage{arydshln}
\usepackage{caption}
\usepackage{subcaption}
\usepackage{xfrac}
\usepackage{etoolbox}
\usepackage{cite}
\usepackage{url}
\usepackage{dcolumn}
\usepackage{hyperref}
\usepackage{courier}
\usepackage{esvect}
\usepackage{commath}
\usepackage{verbatim} % for block comments
\usepackage{enumitem}
\usepackage{hyperref} % for clickable table of contents
\usepackage{braket}
\usepackage{titlesec}
\usepackage{booktabs}
\usepackage{gensymb}
\usepackage{listings}
\usepackage{cancel}
\usepackage[mathscr]{euscript}
\lstset{
    frame=single,
    	basicstyle=\ttfamily\small,
    breaklines=true,
    postbreak=\raisebox{0ex}[0ex][0ex]{\ensuremath{\color{red}\hookrightarrow\space}}
}

% for circled numbers
\usepackage{tikz}
\newcommand*\circled[1]{\tikz[baseline=(char.base)]{
            \node[shape=circle,draw,inner sep=2pt] (char) {#1};}}

\newcommand{\beq}{\begin{equation}}
\newcommand{\eeq}{\end{equation}}
\newcommand{\beqa}{\begin{equation}\begin{aligned}}
\newcommand{\eeqa}{\end{aligned}\end{equation}}

\titleclass{\subsubsubsection}{straight}[\subsection]

% define new command for triple sub sections
\newcounter{subsubsubsection}[subsubsection]
\renewcommand\thesubsubsubsection{\thesubsubsection.\arabic{subsubsubsection}}
\renewcommand\theparagraph{\thesubsubsubsection.\arabic{paragraph}} % optional; useful if paragraphs are to be numbered

\titleformat{\subsubsubsection}
  {\normalfont\normalsize\bfseries}{\thesubsubsubsection}{1em}{}
\titlespacing*{\subsubsubsection}
{0pt}{3.25ex plus 1ex minus .2ex}{1.5ex plus .2ex}

\makeatletter
\renewcommand\paragraph{\@startsection{paragraph}{5}{\z@}%
  {3.25ex \@plus1ex \@minus.2ex}%
  {-1em}%
  {\normalfont\normalsize\bfseries}}
\renewcommand\subparagraph{\@startsection{subparagraph}{6}{\parindent}%
  {3.25ex \@plus1ex \@minus .2ex}%
  {-1em}%
  {\normalfont\normalsize\bfseries}}
\def\toclevel@subsubsubsection{4}
\def\toclevel@paragraph{5}
\def\toclevel@paragraph{6}
\def\l@subsubsubsection{\@dottedtocline{4}{7em}{4em}}
\def\l@paragraph{\@dottedtocline{5}{10em}{5em}}
\def\l@subparagraph{\@dottedtocline{6}{14em}{6em}}
\makeatother


\setcounter{secnumdepth}{4}
\setcounter{tocdepth}{4}
\begin{document}

\title{MATH 228b: HW7, e, f, g}
\author{April Novak}

\maketitle

\section{(a)}
The {\tt euler\_fluxes.m} function returns the Euler fluxes \(F\) given inputs of the four conserved variables - density, two components of momentum, and total energy.

\section{(b)}
The {\tt spectral\_divergence.m} function calculates the divergence of a grid function using the FFT. Given a function \(v(x)\), first \(\hat{v}\) is computed, where \(\hat{v}\) is the Fourier transform of \(v\). Then, because the derivative of a Fourier transform is simply equal to the Fourier transform multiplied by \(ik\), where \(k\) are the discrete frequencies, the derivative in the Fourier transform space can be computed as \(\hat{w}=ik\hat{v}\). Finally, an inverse Fourier transform is performed to obtained \(w\), where \(w\) is the derivative of \(v\). This is the approach used to compute the divergence of a grid function. Note that this function only works for periodic grid functions - otherwise, the divergence ``blows up,'' especially near the domain boundaries.

\section{(c)}
Similar to the use of the Bubnov-Galerkin formulation for convection-diffusion problems, stabilization is required for spectral methods. This is achieved by using a filter, which essentially filters the grid solution by damping out high frequencies by multiplying low frequencies by a number very close to 1.0 and high frequencies by numbers close to 0.0. This is not a conservative method. The spectral filter is implemented by first filtering in the \(x\) direction, and then using this semi-filtered result as input for filtering in the \(y\) direction.

\section{(d)}
The right-hand side of the Euler equations (the divergence of the flux vector) is computed using {\tt euler\_rhs.m} by calling {\tt euler\_fluxes} followed by {\tt spectral\_divergence} once for each of the four equations present.


\begin{comment}
\begin{figure}[H]
\centering
\includegraphics[width=1.0\textwidth]{figures/question2c.png}
\caption{Infinity norm as a function of iteration number for various refinement levels}
\label{fig:question2c}
\end{figure}
\end{comment}








\section{Appendix}
\subsection{{\tt euler\_fluxes.m}}
\lstinputlisting[language=Matlab]{euler_fluxes.m}
\subsection{{\tt spectral\_divergence.m}}
\lstinputlisting[language=Matlab]{spectral_divergence.m}
\subsection{{\tt spectral\_filter.m}}
\lstinputlisting[language=Matlab]{spectral_filter.m}

\end{document}
